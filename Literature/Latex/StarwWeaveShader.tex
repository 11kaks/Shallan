% Reaaliaikainen renderöinti -kurssin harjoitustyö

\documentclass[utf8,bachelor]{gradu3}
\usepackage[bookmarksopen,bookmarksnumbered,linktocpage]{hyperref}
\usepackage{mathtools}
\addbibresource{StarwWeaveShader.bib}


%% ##############################
\begin{document}

\title{Punosvarjostin}

\translatedtitle{Straw weave shader}

%\studyline{}
\avainsanat{avain1, avain2, avain3}
\keywords{avainsanat englanniksi}
\tiivistelma{Tiivistelmä on tyypillisesti 5-10 riviä pitkä esitys työn pääkohdista (tausta, tavoite, tulokset, johtopäätökset).
}
\abstract{Englanninkielinen versio tiivistelmästä.
}

\author{Kimmo Riihiaho}
\contactinformation{\texttt{kimmo.a.riihiaho@student.jyu.fi}}
% jos useita tekijöitä, anna useampi \author-komento
\supervisor{Jarno Kansanaho}
% jos useita ohjaajia, anna useampi \supervisor-komento
\type{} % tämän makron oletus on ''pro gradu -tutkielma'' ja bachelor-optiolla kandidaatintutkielma

\maketitle

\mainmatter

\chapter{Idea}

Tarkoituksena on luoda varjostinohjelma, jolla voi kuvata punosta. Alunperin ajatuksena oli kuvata olkipunosta, mutta varjostin soveltunee myös esim. kankaan kuvaamiseen. Punoksen kuvaaminen $x-y$-tasossa on suhteellisen yksinkertaista, mutta jotta sitä voisi käyttää kaareville pinnoille, tarvitaan koordinaattimuunnoksia varten malliavaruuden lisäksi pinnan paikallinen normaali - luulisin.

Tavoitteena olisi saada varjostin toimimaan edes yhteen suuntaan kaareutuvalle pinnalle.  


\chapter{Matemaattinen perusta}

Yksinkertaisen punosta kuvaavan funktion saa kohtalaisella vaivalla määriteltyä $x-y$-tasoon. Määritellään aluksi skaalausfunktio $f_a$ $x$-akselin suuntaan.

\begin{equation}
\label{eq:warp}
f_a(x) = \frac{xs_o}{s_a} = a,
\end{equation}

jossa $s_o$ on yleisskaalaus, $s_a$ on loimen suuntainen skaalaus ja $x$ on malliavaruuden $x$-koordinaatti. Alaindeksillä $a$ tarkoitetaan loimeen (engl. warp) liittyviä asioita. Loimiskaalauksen $s_a$ kasvaessa loimilangat siirtyvät kauemmaksi toisistaan ja punos loivenee. Yleisskaalauksen $s_o$ (engl. overall) kasvaessa koko punos skaalautuu pienemmäksi.

Vastaavasti määritellään $y$-akselin suuntaan skaalaava funktio $f_e$

\begin{equation}
\label{eq:weft}
f_e(y) = \frac{ys_o}{s_e} = e,
\end{equation}

jossa $s_e$ on kudelangan (engl. weft) leveys.

Seuraavaksi määritellään funktio $f_c$ kuvaamaan kudelangan pyöreyttä

\begin{equation}
\label{eq:curvature}
f_c(e) = r\lvert\sin{e}\rvert,
\end{equation}

jossa kerroin $r$ määrää pyöristyksen voimakkuuden. $r$:n arvolla $0$ saadaan tuotettua kulmikas kude.

Funktio $f_b$ kuvaa kanttiaaltoa (engl. box), jota käytetään vuorottelemaan vierekkäisiä siniaaltoja, jotta ne näyttäisivät erillisiltä kudelangoilta

\begin{equation}
\label{eq:box}
f_b(e) = \pi \lfloor \sin{e} \rfloor.
\end{equation}

Käyttämällä funktioita~\eqref{eq:curvature} ja~\eqref{eq:box} saadaan punoksen yhtälö $f_w$ $a$:n ja $e$:n suhteen 

\begin{equation*} 
\label{eq:weave_ae}
f_w(a,e) = \sin{\big(a + f_b(e)\big)} + f_c(e),
\end{equation*}

joka laajennetaan $x$:n ja $y$:n suhteen syöttämällä sisään yhtälöt~\eqref{eq:warp},~\eqref{eq:weft},~\eqref{eq:curvature} ja~\eqref{eq:box}

\begin{equation}
\label{eq:weave_xy}
f_w(x,y) = \sin{\big(\frac{xs_o}{s_a} + \pi \lfloor \sin{\frac{ys_o}{s_e}} \rfloor\big)} + r\lvert\sin{\frac{ys_o}{s_e}}\rvert.
\end{equation}

Lopuksi on hyvä vielä normalisoida yhtälö min-max -skaalauksella välille $\left[0,1\right]$, jotta sitä voidaan käyttää vaikkapa painokertoimena väreille

\begin{equation}
\label{eq:weave_norm}
f_{wn} = \frac{f_w-\min{(f_w)}}{\max{(f_w)}-\min{(f_w)}}.
\end{equation}

\printbibliography 

\end{document}
